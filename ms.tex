\documentclass[letterpaper,12pt,preprint]{hack_aastex}

% Still to do: estimate the number of 16th mag stars and brighter in the whole
% sky.
% Estimate computation power available.
% Come up with a better target and priorities list.

% Based on the `hogg_nasa` stylesheet by David W. Hogg (NYU)

% Colors.
\usepackage{amsmath}
\usepackage{color}
\usepackage[pagebackref=false]{hyperref}
\definecolor{linkcolor}{rgb}{0,0,0.25}
\hypersetup{colorlinks=true,linkcolor=linkcolor,citecolor=linkcolor,
            filecolor=linkcolor,urlcolor=linkcolor}

\usepackage[para]{footmisc}
\urlstyle{sm}

\usepackage{epsfig}
\usepackage{graphicx}
\setlength{\headheight}{2ex}
\setlength{\headsep}{2ex}

\usepackage{multicol}

\newcommand{\dd}{\,\mathrm{d}}
\newcommand{\bvec}[1]{{\ensuremath{{\boldsymbol{#1}}}}}
\newcommand{\transpose}[1]{{#1}^{\!\mathsf{T}}}
\newcommand{\inverse}[1]{{#1}^{-1}}


%----- exact 1-in margins
% NB: headheight and headsep MUST exist and be set
\setlength{\textwidth}{6.5in}
\setlength{\textheight}{9in}
\addtolength{\textheight}{-1.0\headheight}
\addtolength{\textheight}{-1.0\headsep}
\setlength{\topmargin}{-2em}
\addtolength{\textheight}{-1.0\topmargin}
\setlength{\oddsidemargin}{0.0in}
\setlength{\evensidemargin}{0.0in}

%----- typeset certain kinds of words
%\newcommand{\observatory}[1]{\textsl{#1}}
%\newcommand{\package}[1]{\textsf{#1}}
%\newcommand{\project}[1]{\package{#1}}
%\newcommand{\an}{\package{Astrometry.net}}
%\newcommand{\NASA}{\observatory{NASA}}
%\newcommand{\JWST}{\observatory{JWST}}
%\newcommand{\Kepler}{\observatory{Kepler}}
%\newcommand{\kepler}{\Kepler}
%\newcommand{\Ktwo}{\observatory{K2}}
%\newcommand{\ktwo}{\Ktwo}
%\newcommand{\PLATO}{\observatory{PLATO}}
%\newcommand{\MAST}{\observatory{MAST}}
%\newcommand{\EA}{\observatory{Exoplanet Archive}}
%\newcommand{\TESS}{\observatory{TESS}}
%\newcommand{\galex}{\observatory{GALEX}}
%\newcommand{\Spitzer}{\observatory{Spitzer}}
%\newcommand{\gaia}{\observatory{Gaia}}
%\newcommand{\Gaia}{\gaia}
%\newcommand{\lsst}{\observatory{LSST}}
%\newcommand{\sdss}{\observatory{SDSS}}
%\newcommand{\latin}[1]{\textit{#1}}
%\newcommand{\eg}{\latin{e.g.}}
%\newcommand{\etal}{\latin{et~al.}}
%\newcommand{\etc}{\latin{etc.}}
%\newcommand{\ie}{\latin{i.e.}}
%\newcommand{\vs}{\latin{vs.}}

\newcommand{\doi}[2]{\href{http://dx.doi.org/#1}{\emph{#2}}}
\newcommand{\ads}[2]{\href{http://adsabs.harvard.edu/abs/#1}{\emph{#2}}}
\newcommand{\arxiv}[1]{\href{http://arxiv.org/abs/#1}{arXiv:#1}}

%----- stuff for this proposal in particular.
\newcommand{\kplr}{\package{kplr}}
\newcommand{\PLM}{\package{PLM}}
\newcommand{\OWL}{\package{OWL}}
\newcommand{\George}{\package{George}}
\newcommand{\kpsf}{\package{kpsf}}
\newcommand{\emcee}{\package{emcee}}
\newcommand{\GitHub}{\package{GitHub}}
\newcommand{\KIC}{\textsl{KIC}}
\newcommand{\KOI}{\textsl{KOI}}

%----- typeset journals
% \newcommand{\aj}{Astron.\,J.}
% \newcommand{\apj}{Astrophys.\,J.}
% \newcommand{\apjl}{Astrophys.\,J.\,Lett.}
% \newcommand{\apjs}{Astrophys.\,J.\,Supp.\,Ser.}
% \newcommand{\mnras}{Mon.\,Not.\,Roy.\,Ast.\,Soc.}
% \newcommand{\aap}{Astron.\,\&~Astrophys.}

%----- Tighten up paragraphs and lists
\setlength{\parskip}{0.0ex}
\setlength{\parindent}{0.2in}
\renewenvironment{itemize}{\begin{list}{$\bullet$}{%
  \setlength{\topsep}{0.0ex}%
  \setlength{\parsep}{0.0ex}%
  \setlength{\partopsep}{0.0ex}%
  \setlength{\itemsep}{0.0ex}%
  \setlength{\leftmargin}{1.5\parindent}}}{\end{list}}
\newcounter{actr}
\renewenvironment{enumerate}{\begin{list}{\arabic{actr}.}{%
  \usecounter{actr}%
  \setlength{\topsep}{0.0ex}%
  \setlength{\parsep}{0.0ex}%
  \setlength{\partopsep}{0.0ex}%
  \setlength{\itemsep}{0.0ex}%
  \setlength{\leftmargin}{1.5\parindent}}}{\end{list}}

%----- mess with paragraph spacing!
\makeatletter
\renewcommand\paragraph{\@startsection{paragraph}{4}{\z@}%
                                    {1ex}%
                                    {-1em}%
                                    {\normalfont\normalsize\bfseries}}
\makeatother

%----- Special Hogg list for references
  \newcommand{\hogglist}{%
    \rightmargin=0in
    \leftmargin=0.25in
    \topsep=0ex
    \partopsep=0pt
    \itemsep=0ex
    \parsep=0pt
    \itemindent=-1.0\leftmargin
    \listparindent=\leftmargin
    \settowidth{\labelsep}{~}
    \usecounter{enumi}
  }

%----- side-to-side figure macro
%------- make numbers add up to 94%
 \newlength{\figurewidth}
 \newlength{\captionwidth}
 \newcommand{\ssfigure}[3]{%
   \setlength{\figurewidth}{#2\textwidth}
   \setlength{\captionwidth}{\textwidth}
   \addtolength{\captionwidth}{-\figurewidth}
   \addtolength{\captionwidth}{-0.02\figurewidth}
   \begin{figure}[htb]%
   \begin{tabular}{cc}%
     \begin{minipage}[c]{\figurewidth}%
       \resizebox{\figurewidth}{!}{\includegraphics{#1}}%
     \end{minipage} &%
     \begin{minipage}[c]{\captionwidth}%
       \textsf{\caption[]{\footnotesize {#3}}}%
     \end{minipage}%
   \end{tabular}%
   \end{figure}}

 \newcommand{\ssfiguretwo}[4]{%
   \setlength{\figurewidth}{#3\textwidth}
   \setlength{\captionwidth}{\textwidth}
   \addtolength{\captionwidth}{-\figurewidth}
   \addtolength{\captionwidth}{-0.02\figurewidth}
   \begin{figure}[htb]%
   \begin{tabular}{cc}%
     \begin{minipage}[c]{\figurewidth}%
       \resizebox{\figurewidth}{!}{\includegraphics{#1}}\\
       \resizebox{\figurewidth}{!}{\includegraphics{#2}}%
     \end{minipage} &%
     \begin{minipage}[c]{\captionwidth}%
       \textsf{\caption[]{\footnotesize {#4}}}%
     \end{minipage}%
   \end{tabular}%
   \end{figure}}

%----- top-bottom figure macro
 \newlength{\figureheight}
 \setlength{\figureheight}{0.75\textheight}
 \newcommand{\tbfigure}[2]{%
   \begin{figure}[htp]%
   \resizebox{\textwidth}{!}{\includegraphics{#1}}%
   \textsf{\caption[]{\footnotesize {#2}}}%
   \end{figure}}

%----- deal with pdf page-size stupidity
\special{papersize=8.5in,11in}
\setlength{\pdfpageheight}{\paperheight}
\setlength{\pdfpagewidth}{\paperwidth}

% no more bad lines!
\sloppy\sloppypar

% A better underline!
% tex.stackexchange.com/questions/36894/underline-omitting-the-descenders
\usepackage[T1]{fontenc}
\usepackage[latin1]{inputenc}
\usepackage{soul}
% \usepackage{xcolor}
\usepackage{xparse}
\makeatletter

% https://tex.stackexchange.com/questions/36894/underline-omitting-the-descenders
\ExplSyntaxOn
\cs_new:Npn \white_text:n #1
  {
    \fp_set:Nn \l_tmpa_fp {#1 * .01}
    \llap{\textcolor{white}{\the\SOUL@syllable}\hspace{\fp_to_decimal:N \l_tmpa_fp em}}
    \llap{\textcolor{white}{\the\SOUL@syllable}\hspace{-\fp_to_decimal:N \l_tmpa_fp em}}
  }
\NewDocumentCommand{\whiten}{ m }
    {
      \int_step_function:nnnN {1}{1}{#1} \white_text:n
    }
\ExplSyntaxOff

\NewDocumentCommand{ \varul }{ D<>{5} O{0.2ex} O{0.1ex} +m } {%
\begingroup
\setul{#2}{#3}%
\def\SOUL@uleverysyllable{%
   \setbox0=\hbox{\the\SOUL@syllable}%
   \ifdim\dp0>\z@
      \SOUL@ulunderline{\phantom{\the\SOUL@syllable}}%
      \whiten{#1}%
      \llap{%
        \the\SOUL@syllable
        \SOUL@setkern\SOUL@charkern
      }%
   \else
       \SOUL@ulunderline{%
         \the\SOUL@syllable
         \SOUL@setkern\SOUL@charkern
       }%
   \fi}%
    \ul{#4}%
\endgroup
}
\makeatother

% http://www.guitex.org/home/it/forum/5-tex-e-latex/83195-la-libreria-hobby-tikz-non-funziona-piu#83203
%\ExplSyntaxOn
%\cs_if_exist:NF \prg_stepwise_function:nnnN { \cs_gset_eq:NN \prg_stepwise_function:nnnN \int_step_function:nnnN }
%\cs_if_exist:NF \prg_stepwise_inline:nnnn { \cs_gset_eq:NN \prg_stepwise_inline:nnnn \int_step_inline:nnnn }
%\ExplSyntaxOff

%\ExplSyntaxOn
%\cs_new:Npn \white_text:n #1
%  {
%    \fp_set:Nn \l_tmpa_fp {.01}
%    \fp_mul:Nn \l_tmpa_fp {#1}
%    \llap{\textcolor{white}{\the\SOUL@syllable}\hspace{\fp_to_decimal:N \l_tmpa_fp em}}
%    \llap{\textcolor{white}{\the\SOUL@syllable}\hspace{-\fp_to_decimal:N \l_tmpa_fp em}}
%  }
%\NewDocumentCommand{\whiten}{ m }
%    {
%      \prg_stepwise_function:nnnN {1}{1}{#1} \white_text:n
%      % \int_step_function:nnnN {1}{1}{#1} \white_text:n
%    }
%\ExplSyntaxOff

%\NewDocumentCommand{ \varul }{ D<>{5} O{0.2ex} O{0.1ex} +m } {%
%\begingroup
%\setul{#2}{#3}%
%\def\SOUL@uleverysyllable{%
%   \setbox0=\hbox{\the\SOUL@syllable}%
%   \ifdim\dp0>\z@
%      \SOUL@ulunderline{\phantom{\the\SOUL@syllable}}%
%      \whiten{#1}%
%      \llap{%
%        \the\SOUL@syllable
%        \SOUL@setkern\SOUL@charkern
%      }%
%   \else
%       \SOUL@ulunderline{%
%         \the\SOUL@syllable
%         \SOUL@setkern\SOUL@charkern
%       }%
%   \fi}%
%    \ul{#4}%
%\endgroup
%}
%\makeatother


% Underline paragraph titles
\usepackage[explicit]{titlesec}
\titleformat{\paragraph}[runin]
    {\normalfont\normalsize\bfseries}{\theparagraph}{1em}
    {\varul{#1}}

\pagestyle{myheadings}
\markright{\textsf{\footnotesize %
    Investigating rotational evolution with TESS field dwarfs}}

\bibliographystyle{aasjournal}

\newcommand{\Kepler}{{\it Kepler}}
\newcommand{\kepler}{\Kepler}
\newcommand{\corot}{{\it CoRoT}}
\newcommand{\Ktwo}{{\it K2}}
\newcommand{\ktwo}{\Ktwo}
\newcommand{\TESS}{{\it TESS}}
\newcommand{\tess}{{\it TESS}}
\newcommand{\LSST}{{\it LSST}}
\newcommand{\Wfirst}{{\it Wfirst}}
\newcommand{\SDSS}{{\it SDSS}}
\newcommand{\PLATO}{{\it PLATO}}
\newcommand{\Gaia}{{\it Gaia}}
\newcommand{\gaia}{{\it Gaia}}
\newcommand{\RAVE}{{\it RAVE}}
\newcommand{\rave}{{\it RAVE}}
\newcommand{\hermes}{{\it HERMES}}
\newcommand{\HERMES}{{\it HERMES}}
\newcommand{\Teff}{$T_{\mathrm{eff}}$}
\newcommand{\teff}{$T_{\mathrm{eff}}$}
\newcommand{\FeH}{[Fe/H]}
\newcommand{\feh}{[Fe/H]}
\newcommand{\ie}{{\it i.e.}}
\newcommand{\eg}{{\it e.g.}}
\newcommand{\logg}{log \emph{g}}
\newcommand{\dnu}{$\Delta \nu$}
\newcommand{\numax}{$\nu_{\mathrm{max}}$}

\newcommand{\racomment}[1]{{\color{red}#1}}

\newcommand{\columbia}{1}
\newcommand{\ww}{2}
\newcommand{\cca}{3}
\newcommand{\florida}{4}
\newcommand{\princeton}{5}
\newcommand{\nsf}{4}
\newcommand{\simons}{2}
\newcommand{\hubble}{7}

\begin{document}

\title{Investigating magnetic dynamo evolution with TESS field dwarfs}

Of the measurable properties for a large ensemble of field stars, rotation
periods contain the most information about stellar age, and provide the best
leverage for advancing our knowledge of galactic archeology as well as
exoplanet population demographics via gyrochronology.
Angular momentum is carried away though magnetically driven stellar winds,
which slows the star's rotation over cosmic time.
This rotation-based `clock' is known as gyrochronology.
Cool spots on the star's surface rotate in-to and out-of view, creating small
amplitude ($\pm$ 1\%) quasi-periodic changes in the stellar brightness.
Based on results from \Kepler, using precise light curves available from the
\TESS\ mission, we expect to measure the rotation periods of around 20\% of FG
stars and 80\% of KM stars in the Candidate Target List (CTL)\footnote{The
difference between these percentages comes from a dramatic increase in
magnetic activity towards lower masses.}.
For many of these stars it will be possible to infer an age from their
rotation periods via gyrochronology.

Some outstanding mysteries regarding the nature of magnetic braking have been
revealed, but thus-far unanswered by, \kepler.
Firstly, a mysterious gap in the rotation period distribution of \Kepler\
dwarfs requires either a sharp transition in magnetic dynamo geometry or a gap
in the local star formation history as an explanation \citep{mcquillan2014,
davenport2017}.
Secondly, a transitioning magnetic dynamo appears to be responsible for
inefficient magnetic braking at old ages in Solar-mass stars.
{\bf We propose to help answer these questions by providing a catalog of
rotation periods for \TESS\ Full Frame Image (FFI) stars.}
% We also intend to use gyrochronology in the cases where it {\it is} applicable
% to learn about the evolution of exoplanets across the galaxy.

\paragraph{Scientific Justification}

The bimodal period distribution among field stars discovered by
\citet{mcquillan2013}, is shown in Figure \ref{fig:davenport}.
Recently, Co-I Davenport discovered this period bimodality extends throughout
all masses in the Kepler rotation sample for nearby stars
\citep{davenport2017}.
This feature could either reflect a drop in the star formation rate around 600
Myr ago or could be explained by a previously unknown variation in the
spin-down evolution for low-mass stars.
The TESS FFI and CTL targets will provide the ideal dataset to test these two
scenarios explaining the appearance of a period bimodality.
If the bimodal period distribution reflects a discontinuous age distribution,
the feature should be local and may disappear at greater distances or along
different lines of sight.
However, if the bimodality is truly due to a transition point in the spin-down
evolution at young ages, there should be little to no variation in the feature
with galactic position.
We will match the TESS FFI and CTL targets to the upcoming data
release from the Gaia mission \citep{perryman2001} to map the rotation period
distribution as a function of galactocentric position.
For the all-sky analysis we will focus on F and G stars since the 27.5 day
baseline of \TESS\ fields will limit measureable rotation periods to less than
around 15 days.
With the April 2018 data release from Gaia (DR2), we estimate that we will be
able to study rotation periods for G dwarfs in the TESS FFIs out to $\sim$3
kpc.
To characterize the rotation period bimodality as a function of galactic
position, we propose to analyse the light curves of 10,000 FFI and two-minute
targets that overlap with the Tycho-Gaia Astrometric Solution (TGAS) catalog,
selecting FGKM dwarfs with a range of galactic positions.
The FFI targets will be essential in order to reach $\sim$3kpc distances.
% Of these, the stars lying in the CVZs will be our highest priority.
M dwarfs rotation periods in the CVZz will allow us to study the rotation
period bimodality at lower masses, since the bimodality appears at longer
rotation periods for lower mass stars.
Unlike \kepler\ and \ktwo\ light curves, \TESS\ provides the all-sky coverage
necessary to map the period bimodality across the galaxy.

\ssfigure{Davenport.png}{0.5}{%
From Davenport (2017).
    Rotation period as a function of effective
temperature for the full McQuillan et al., (2014) Kepler sample in black, and
the subset of these stars that also feature in the TGAS Gaia DR1 catalogue in
blue. Contaminating giants have been removed from the blue sample and the
rotation period bimodality is revealed to exist across all temperatures shown.
    The red line is a 600 Myr rotational isochrone (also known as a gyrochrone).
\label{fig:davenport}
}

Although the classical spin-down law of \citet{skumanich1972}: Period
$\propto$ Age$^1/2$ holds for all open clusters with measured periods, it does
not appear to describe old field stars \citep{angus2015, van-saders2016}.
\citet{van-saders2016} found that including a transition to a weakened magnetic
braking regime in the gyrochronology models, at a critical Rossby number,
provides an improved fit to the data.
Further calibration is needed however; a lack of rotation periods and reliable
ages for old and low-mass stars leaves the rotational evolution of stars older
than the Sun relatively unexplored.
\TESS\ will provide thousands of new rotation periods for both FFI and
two-minute cadence stars, with the maximum measurable period strongly
depending on the length of time it is observed for.
Of particular interest are the potentially long rotation periods measurable
for stars that fall in the Continuous Viewing Zones (CVZs).
Many of these will have spectra from the \TESS --\HERMES\ survey, from which it
will be possible to calculate isochronal ages, enabling further calibration of
the gyrochronology relations.
We will also use the rotation periods of TESS field dwarfs over the whole sky
to test the gyrochronology relations using galactic kinematics.
Many stars observable by \tess\ will have proper motions, parallaxes,
positions and radial velocities published in the second \Gaia\ data release.
These parameters provide the information necessary to calculate galactocentric
positions and action angles of the stars, both of which are age indicators.
% PI Angus is currently using the \kepler\ rotation periods of stars in the
% first \Gaia\ data release to calibrate the relations between rotation period
% and vertical action dispersion, two tracers of age.
% Since this relation is likely vary with both galactic longitude and latitude,
% the rotation periods produced from \TESS\ light curves will enable a
% comprehensive, all-sky calibration of these relations.
Since one of the few ways to accurately age-date fully convective, late M
dwarfs is via kinematics, these new relations will help to infer ages for all
stars with M$<$ 0.35M$_\odot$, to which gyrochronology cannot be applied.
% In order to calibrate gyrochronology using rotation periods of FGK dwarfs with
% \tess--\hermes\ spectra we propose to prioritize rotation periods of stars
% in the CVZs with spectra, followed by those without.

In addition, we will use the rotation periods of comoving stars identified in
the first \Gaia\ data release \citep{oh2016} to quantify the accuracy and
precision of gyrochronology.
We have identified 38 stars in 19 pairs with \Gaia\ proper motions and radial
velocities from \RAVE\ that are observable by \TESS, using {\tt tvguide}.
Of these, 34 stars in 17 pairs are already in the CTL list.
Because the stars in these pairs are assumed to be recently disrupted
binaries, we expect them to have the same age and measuring their rotation
periods will therefore provide a test of gyrochronology.

% Finally, we hope to use the improved gyrochronology relations, calibrated
% using \TESS\ data to infer the dependence of planet frequency on age and
% position in the galaxy.
% \kepler\ revealed tantalizing hints that exoplanets on short orbital periods
% have larger radii and are more infrequent around young stars \citep{mann2017a,
% mann2017b, rizzuto2017}.
% Do these trends continue into older ages and do they exist in the field?
% We will answer these questions by measuring rotation periods for all \TESS\
% field stars.

\paragraph{Technical Feasibility}
The extraction of a rotation period from a light curve can be as simple as
computing a Lomb-Scargle periodogram or, in cases where the signal is less
clear, can be inferred via modeling the correlation between data points.
This latter approach could involve either computing an autocorrelation
function \citep{mcquillan2013}, or fitting a Gaussian process to the time
series \citep{angus2017, foreman-mackey2017}.
In cases where signals have long periods and low amplitudes, care is needed to
separate real astrophysical signal from instrumental systematics.
The measurement of rotation periods is less sensitive to crowding and source
confusion than exoplanet transit characterization because the rotation period
is not effected by photometric dilution.
We will apply two complementary methods to extract and calibrate light curves
from the TESS FFIs.
First, for bright or isolated targets, we will follow \citet{montet2017} to
estimate aperture shapes and perform aperture photometry for bright sources.
Using an ensemble of sources, we will de-trend these light curves using a
modified version of \textsf{everest} \citep{luger2016, luger2017} designed to
preserve photometric signatures of rotation.
This will be achieved by fitting for the systematic effects in the light curve
using the \textsf{everest} model simultaneously with a Gaussian Process model
for the astrophysical variation.
Both \citet{aigrain2016} and \citet{luger2016} demonstrated that this can
preserve stellar variability signals and we will use the \textsf{celerite}
algorithm \citep{dfm2017} to scale the computations to the size of TESS FFI
datasets.
The precision of existing photometric de-trending methods degrades in crowded
fields \citep[for example,][]{luger2017}.
However, to make robust measurements of rotation periods, we do not need
absolutely calibrated light curves.
Therefore, in crowded fields, we will apply a difference imaging method that
was developed for the K2 Campaign 9 microlensing project \citep{henderson2016}
based on the \textsf{CPM} \citep{wang2016} to robustly measure the photometric
variations of crowded sources.
Unlike standard difference imaging methods, this procedure does not require a
reference image.
Instead, a causal data-driven model is built to predict the time series in
every pixel taking systematic effects into account and the residuals between
the observations and the model predictions provide an estimate of the
astrophysical variability in each pixel.
We will tune this method preserve rotation signals and apply it to detect
rotation across the FFIs.

The full set of available \TESS\ FFI and two-minute cadence light curves will
be exceedingly large and we intend to prioritize certain groups of stars for
period analysis, calculating a simple (and computationally fast) Lomb-Scargle
periodogram for the low priority objects and running a full probabilistic
period analysis on the high priority targets.
High priority targets include: comoving pairs, stars in the CVZs (especially
those with spectra), stars in TGAS, stars at large distances, plus any
potential asteroseismic dwarfs and exoplanet hosts.
Based on the expected photometric precision of \tess\ \citep{sullivan2015}, we
expect to be able to measure rotation periods for FFI stars down to 16th
magnitude.
We note, however that not all stars will have measurable rotation periods,
even if they are bright enough --- not all stars show rotational variability
in their light curves either because they are pole-on or because they are
inactive.
% Of these, the stars that also lie in the CVZs as well as the identified \Gaia\
% comoving pairs will be our highest priority.
% Our next priority will be stars in the CVZs, with spectra, that do not appear
% in TGAS, followed by those without spectra.
% M dwarfs rotation periods in the CVZz will allow us to study the rotation
% period bimodality at lower masses, since the bimodality appears at longer
% rotation periods for lower mass stars.
% We will then focus on measuring rotation periods for a sample of stars that is
% representative of the full set of FFI and CTL targets in order to enable
% exoplanet population studies.
% An large amount of computational resources are available to us, with access to
% the Columbia supercomputer as well as facilities at the Flatiron Institute.
% We expect that each target will require around one minute of computation time
% and have access to around 10,000 computing hours.
% We therefore expect to analyze around 60,000 light curves.

We have already begun to construct a rotation period catalog of \ktwo\ stars,
for which we have received funding from the Archival Data Analysis Program
(ADAP).
Figure \ref{fig:kalesalad} shows a map of preliminary rotation periods for one
of four \ktwo\ fields processed so far.

\ssfigure{Kalesalad.png}{0.5}{%
All stars observed during K2 campaign 4, plotted according to their
equatorial coordinates and colored by their preliminary rotation period.
These rotation periods were measured using a simple ACF method, applied to
    everest (Luger et al., 2015) light curves.
\label{fig:kalesalad}
}

\paragraph{Expected Impact}
Initially we will provide rotation period posterior probability density
functions and best fit estimates for around 30,000 high priority two-minute
cadence and FFI targets.
We will then produce a catalog of rotation periods for lower priority stars
using a combination of Lomb-Scargle periodogram and autocorrelation
techniques.
Finally, we will release a catalog of rotation periods for two-minute cadence
targets.
We will also release an open source python package for measuring rotation
periods of both FFI and two-minute cadence targets.

\paragraph{Budget Justification}
PI Angus intends to use the budget to employ a student for 3-4 months.
In this time the student will assist in extracting light curves from the FFIs,
measuring rotation periods and building a rotation period catalog.
The student will also be involved in the scientific project of their choosing:
either the rotation period bimodality, gyrochronology of low-mass stars, or
gyrochronology of comoving pairs.
The remaining budget will be used for traveling to meet with other Co-Is in
order to perform this analysis.
\paragraph{References}

{\footnotesize
\bibliography{references}
}

\end{document}
